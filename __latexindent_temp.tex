\subsection{Specification}
\begin{enumerate}
\item Accident Download System
\begin{enumerate}
    \item The System must be capable of interfacing with the TFL API to download accidents
    \item The System must be capable of parsing accident data to determine which accidents are relevant
    \item The System must be capable of storing accident data in a easily accessible file for the route finding algorithm to parse
\end{enumerate} 
\item Route Finding Systems
\begin{enumerate}
    \item  The System should be capable of processing an Open Street Map map file into a suitable data structure
    \item The System should be capable of processing the accident data from the Accident Download System
    \item The System should be capable of attaching the data from the Accident Download System to the Suitable Data Structure from specification point \textbf{2.1}
    \item The System should be capable of using Accident Data combined with other suitable cost estimation functions to find a directed route between two points that are places on roads in London
    \item The System should always suggest a route that can be followed while observing all currently known laws of physics
    \item The System should always suggest a route that can be followed while observing all international, national, and local laws
    \item The System should suggest a safe route wherever possible
    \item The System should have guards in place for certain types of roads which are deemed too dangerous to consider
    \item The System should be able to generate this route quickly
    \item The System should have a mode for preprocessing to make the route generation even quicker
    \item When using the same parameters, the route generated using a preprocessed graph and associated algorithms should be the same as that generated by the non-preprocessing based approach
    \item The System should be easy to use
    \item The System should not crash, and should give appropriate non crashing errors if user data is determined to be bad
    \item The parameters for the cost estimation functions should be user definable, but sensible defaults should also be defined
\end{enumerate}
\end{enumerate}
\newpage
\section{Design}
\subsection{Accident Download System}
As outlined in the Specification, this system should be capable of interfacing with the TFL API, downloading the accidents, deciding which ones are relevant, and storing this in a useful format.
\subsubsection{Pulling from the TFL API}
The Transport For London API is an excellent API which can provide information on many different aspects of the Transport For London network. The specific API which I used is the AccidentStats API.
This API is very simple, you simply request a given year and a JSON object is returned which contains all of the accidents that happened in london that year. These consist of all the accidents that were reported tocdepth
the police as happening in that year. Of course, there will be many more accidents than are on the API, but these will mostly be more minor accidents.
The Data is returned as a list of accidents, formatted as shown in Figure \ref{Default}.
\begin{figure}[t]
\begin{lstlisting}[language=Python]
[{
    "id": 0,
    "lat": 0.0,
    "lon": 0.0,
    "location": "string",
    "date": "string",
    "severity": "string",
    "borough": "string",
    "casualties": [{
        "age": 0,
        "class": "string",
        "severity": "string",
        "mode": "string",
        "ageBand": "string"
    }],
    "vehicles": [{
        "type": "string"
    }]
}]
    \end{lstlisting}
\caption{The Default output from the API}
\label{Default}
\end{figure}
\\
As the data about accidents that happened in the past is not going to change any time soon, and TFL only updates this API every year, it is simplest just to download the files once, parse it once, and then 
use that result in the route finder.\\
TFL started gathering this data in 2005, and the most recent update was in 2019, so I wrote a simple python script to download all the data from 2005 to 2019 and save it in a subfolder called \texttt{accidents}.
The program loops through all the years between 2005 and 2019 and fills a JSON file for that year. \\
\subsubsection{Parsing the Data}
The next step was to go through the data from all the years, and save all the accidents that were pertinent to my project. As seen in Figure \ref{Default}, each accident listing has information on casualties,\\
of which there may be many. As this is a cycling application, I only wanted data on accidents where at least one of the casualties was a cyclist. The API provides a lot of data, but all that I decided was relevant was
the latitude and longitude, as well as the severity of the incident. All of this data was ultimately saved to a file called \texttt{output.json}.
 The code used for parsing is shown in Figure \ref{Download.py}.
\begin{figure}[t]
    \begin{lstlisting}[language=Python]
import json
accidents = []
for year in range(2005,2020):
    print(len(accidents), year)
    with open(f"accidents/{year}.json","r") as outfile:
        data = json.load(outfile)
    for x in data:
        if len([person for person in x["casualties"] if person["mode"] == "PedalCycle"]) >= 1:
            accidents.append([x["lat"],x["lon"],x["severity"]])
with open("output.json","w") as outfile:
    json.dump(accidents,outfile)
    \end{lstlisting}
\caption{The script I wrote to parse the data}
\label{Download.py}
\end{figure}
\label{end}